\documentclass[twoside,11pt]{article}

% Any additional packages needed should be included after jmlr2e.
% Note that jmlr2e.sty includes epsfig, amssymb, natbib and graphicx,
% and defines many common macros, such as 'proof' and 'example'.
%
% It also sets the bibliographystyle to plainnat; for more information on
% natbib citation styles, see the natbib documentation, a copy of which
% is archived at http://www.jmlr.org/format/natbib.pdf

\usepackage{csc498project}



% Short headings should be running head and authors last names

\ShortHeadings{Atari Breakout Reinforcement Learning Environment}{Your names}
\firstpageno{1}


\title{Atari Breakout Reinforcement Learning Environment}

\author{\name Haider Sajjad \email haider.sajjad@mail.utoronto.ca \\
       \addr 1004076251\\
       \AND
      \name Weiyu Li \email weiyu.li@mail.utoronto.ca \\
       \addr 1003765981}

\begin{document}

\maketitle

\begin{abstract}%   <- trailing '%' for backward compatibility of .sty file
\textit{
Atari Breakout environment implementation and training an agent using multiple algorithms over a generated environment (changing brick layouts)}

\end{abstract}

\section{Introduction}
Our project is implementing an Atari breakout environment and training an agent to play it across general levels. The project repository can be found here: https://github.com/duoduocai-dot/csc498-project

\subsection{asldkfjasf}
sdklfklasdf

\section{Motivation and Impact}

\section{Intuition}

\bibliography{bibliography}

\end{document}